\chapter{Introduction}
Putting in a citation so that the bibliography file is used\cite{ex}.

\section{Overview}
Species diversity is an important measurement of ecological communities. Scientists 
believe that there is relationship between species diversity and ecosystem processes \cite{Loreau:2001aa}. Evaluating the species diversity in a community is a central research topic 
in classic ecology. Many methods have been developed since decades ago, which make the 
question like "how many species of birds in this habitat" easier to answer. 
Nevertheless, scientists have not started to think seriously about other old questions like 
"How many species are there on earth?" \cite{May:1988aa} or "How many species are there in the ocean?" \cite{Mora:2011aa} until
not long time ago.
Why? The answer is simple: Microorganisms represet the vast majority of the Earth's biodiversity
and the assessment of microbial diversity is simply hard. Actually it is believed that 
microbial diversity is the outermost frontier of the exploration of diversity \cite{magurran2011biological}.
Microorganisms are ubiquitous. They were the first forms of life on the Earth. 
There are more bacterial cells inhabiting our body than our own cells \cite{Savage:1977aa}.
They are essential to all life, too. There are several reasons why assessment of microbial diversity
is such a challenge. The concept of species is ambiguous. Morphological examination is impossible.
95\% or more of the microbial diversity in the biosphere can not be cultivated with standard 
culturing techniques \cite{Curtis:2002aa}. To tackle these obstacles, metagenomics emerges, with
 the boost given by the progress of next generation sequencing technology. Lots of metagenomics 
 projects have been performed on samples from acid mine drainage channels to human gut. For 
 complex samples like soil, the resulting data set will be huge. There are approximately a billion 
 microbial cells, with about 4 petabase pairs of DNA($4*{10}^{12}$ bp). Since we have limited
 sequencing power and other financial strain, the resulting metagenomics data set from 
 high diversity sample like soil only corresponds to a tiny fraction of the actual genomic
 content in the sample. The large size of data set and the low coverage make the assessment of microbial 
 diversity of high diversity sample even harder. Novel method is highly needed. 

\section{Problem Statement}

\section{Significance of Research}

\section{Outline of Dissertation}


